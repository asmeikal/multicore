\documentclass[landscape, a4paper, draft]{article}

\usepackage[italian]{babel}
\usepackage{fullpage}
\usepackage{parskip}
\usepackage{subcaption} 
\usepackage{pgfplots}
\pgfplotsset{compat=1.5}
\usetikzlibrary{shapes,arrows,positioning,automata}

\begin{document}

Nei grafici del tempo di esecuzione abbiamo inserito una iperbole a indicare l'andamento ideale del tempo di esecuzione all'aumentare del numero di thread.
La linea orizzontale rappresenta il ``limite'' teorico di $\frac{T_1}{p}$, con $T_1$ a indicare il tempo sequenziale e $p$ il numero di processori fisici.
Con hyperthreading attivo, questo limite viene superato.

Nei grafici dello speedup abbiamo disegnato la bisettrice del quadrante, a indicare lo speedup perfettamente lineare.
Il limite \`e rappresentato anche qui dalla linea orizzontale $y = p$, il massimo speedup che si pu\`o sperare di ottenere con $p$ processori fisici.
Anche questo limite viene superato dall'hyperthreading.

\listoffigures


\begin{figure}
\centering
\caption{Speedup/thread e tempo/thread della prima implementazione, contro password di lunghezza 4, sul Raspberry Pi.}
\begin{subfigure}[t]{0.3\textwidth}
\resizebox{\textwidth}{!}{
	\begin{tikzpicture}
		\begin{axis}[
			scale only axis,
			width=\textwidth,
			xtick=data,
			domain=1:8,
			ymajorgrids,
			xlabel={\# thread},
			ylabel={speedup},
        ]
			\pgfplotstableread{tests/pi_prima_2_41.dat}{\pimone}
			\pgfplotstableread{tests/pi_prima_2_42.dat}{\pimtwo}
			\pgfplotstableread{tests/pi_prima_2_43.dat}{\pimthree}
			\addplot[mark=o,blue] 
			 table[x=thread,y=speedup]  {\pimone};
			 \addlegendentry{Work unit = 1}
			\addplot[mark=o,red] 
			 table[x=thread,y=speedup]  {\pimtwo};
			 \addlegendentry{Work unit = 2}
			\addplot[mark=o,green]
			 table[x=thread,y=speedup]  {\pimthree};
			 \addlegendentry{Work unit = 3}
			\addplot[dashed,blue] {x};
			\addplot[dashed] {4};
			\node[above] at (axis cs:2,4) {\# processori};
		\end{axis}
	\end{tikzpicture}
	}
	\caption{JVM Oracle (1.7)}
\end{subfigure}
\qquad
\begin{subfigure}[t]{0.3\textwidth}
\resizebox{\textwidth}{!}{
	\begin{tikzpicture}
		\begin{axis}[
			scale only axis,
			width=\textwidth,
			xtick=data,
			domain=1:8,
			ymajorgrids,
			xlabel={\# thread},
			ylabel={speedup},
        ]
			\pgfplotstableread{tests/pi_prima_4_41.dat}{\pimone}
			\pgfplotstableread{tests/pi_prima_4_42.dat}{\pimtwo}
			\pgfplotstableread{tests/pi_prima_4_43.dat}{\pimthree}
			\addplot[mark=o,blue] 
			 table[x=thread,y=speedup]  {\pimone};
			 \addlegendentry{Work unit = 1}
			\addplot[mark=o,red] 
			 table[x=thread,y=speedup]  {\pimtwo};
			 \addlegendentry{Work unit = 2}
			\addplot[mark=o,green]
			 table[x=thread,y=speedup]  {\pimthree};
			 \addlegendentry{Work unit = 3}
			\addplot[dashed,blue] {x};
			\addplot[dashed] {4};
			\node[above] at (axis cs:2,4) {\# processori};
		\end{axis}
	\end{tikzpicture}
	}
	\caption{JVM Oracle (1.8)}
\end{subfigure}
\qquad
\begin{subfigure}[t]{0.3\textwidth}
\resizebox{\textwidth}{!}{
	\begin{tikzpicture}
		\begin{axis}[
			scale only axis,
			width=\textwidth,
			xtick=data,
			domain=1:8,
			ymajorgrids,
			xlabel={\# thread},
			ylabel={speedup},
        ]
			\pgfplotstableread{tests/pi_prima_3_41.dat}{\pimone}
			\pgfplotstableread{tests/pi_prima_3_42.dat}{\pimtwo}
			\pgfplotstableread{tests/pi_prima_3_43.dat}{\pimthree}
			\addplot[mark=o,blue] 
			 table[x=thread,y=speedup]  {\pimone};
			 \addlegendentry{Work unit = 1}
			\addplot[mark=o,red] 
			 table[x=thread,y=speedup]  {\pimtwo};
			 \addlegendentry{Work unit = 2}
			\addplot[mark=o,green]
			 table[x=thread,y=speedup]  {\pimthree};
			 \addlegendentry{Work unit = 3}
			\addplot[dashed,blue] {x};
			\addplot[dashed] {4};
			\node[above] at (axis cs:2,4) {\# processori};
		\end{axis}
	\end{tikzpicture}
	}
	\caption{JVM OpenJDK}
\end{subfigure}
\\
\begin{subfigure}[b]{0.3\textwidth}
\resizebox{\textwidth}{!}{
	\begin{tikzpicture}
		\begin{axis}[
			scale only axis,
			width=\textwidth,
			xtick=data,
			ymajorgrids,
			ymin=0.1,
			xlabel={\# thread},
			ylabel={tempo ($s$)},
		]
			\pgfplotstableread{tests/pi_prima_2_41.dat}{\pimone}
			\pgfplotstableread{tests/pi_prima_2_42.dat}{\pimtwo}
			\pgfplotstableread{tests/pi_prima_2_43.dat}{\pimthree}
			\addplot[mark=o,blue] table[x=thread,y=tempo] {\pimone};
			 \addlegendentry{Work unit = 1}
			\addplot[mark=o,red] table[x=thread,y=tempo] {\pimtwo};
			 \addlegendentry{Work unit = 2}
			\addplot[mark=o,green] table[x=thread,y=tempo] {\pimthree};
			 \addlegendentry{Work unit = 3}
			\pgfplotstablegetelem{0}{tempo}\of{\pimone};
			\pgfmathsetmacro{\yn}{\pgfplotsretval};
    		\addplot[domain=1:8,dashed,samples=400] (x,{\yn/x});
			\addplot[domain=1:8,dashed] {\yn/4};
  		\end{axis}
	\end{tikzpicture}
	}
	\caption{JVM Oracle (1.7)}
\end{subfigure}
\qquad
\begin{subfigure}[b]{0.3\textwidth}
\resizebox{\textwidth}{!}{
	\begin{tikzpicture}
		\begin{axis}[
			scale only axis,
			width=\textwidth,
			xtick=data,
			ymajorgrids,
			ymin=0.1,
			xlabel={\# thread},
			ylabel={tempo ($s$)},
		]
			\pgfplotstableread{tests/pi_prima_4_41.dat}{\pimone}
			\pgfplotstableread{tests/pi_prima_4_42.dat}{\pimtwo}
			\pgfplotstableread{tests/pi_prima_4_43.dat}{\pimthree}
			\addplot[mark=o,blue] table[x=thread,y=tempo] {\pimone};
			 \addlegendentry{Work unit = 1}
			\addplot[mark=o,red] table[x=thread,y=tempo] {\pimtwo};
			 \addlegendentry{Work unit = 2}
			\addplot[mark=o,green] table[x=thread,y=tempo] {\pimthree};
			 \addlegendentry{Work unit = 3}
			\pgfplotstablegetelem{0}{tempo}\of{\pimone};
			\pgfmathsetmacro{\yn}{\pgfplotsretval};
    		\addplot[domain=1:8,dashed,samples=400] (x,{\yn/x});
			\addplot[domain=1:8,dashed] {\yn/4};
  		\end{axis}
	\end{tikzpicture}
	}
	\caption{JVM Oracle (1.8)}
\end{subfigure}
\qquad
\begin{subfigure}[b]{0.3\textwidth}
\resizebox{\textwidth}{!}{
	\begin{tikzpicture}
		\begin{axis}[
			scale only axis,
			width=\textwidth,
			xtick=data,
			ymajorgrids,
			ymin=0.1,
			xlabel={\# thread},
			ylabel={tempo ($s$)},
		]
			\pgfplotstableread{tests/pi_prima_3_41.dat}{\pimone}
			\pgfplotstableread{tests/pi_prima_3_42.dat}{\pimtwo}
			\pgfplotstableread{tests/pi_prima_3_43.dat}{\pimthree}
			\addplot[mark=o,blue] table[x=thread,y=tempo] {\pimone};
			 \addlegendentry{Work unit = 1}
			\addplot[mark=o,red] table[x=thread,y=tempo] {\pimtwo};
			 \addlegendentry{Work unit = 2}
			\addplot[mark=o,green] table[x=thread,y=tempo] {\pimthree};
			 \addlegendentry{Work unit = 3}
			\pgfplotstablegetelem{0}{tempo}\of{\pimone};
			\pgfmathsetmacro{\yn}{\pgfplotsretval};
    		\addplot[domain=1:8,dashed,samples=400] (x,{\yn/x});
			\addplot[domain=1:8,dashed] {\yn/4};
  		\end{axis}
	\end{tikzpicture}
	}
	\caption{JVM OpenJDK}
\end{subfigure}
\end{figure}

\clearpage

\begin{figure}
\centering
\caption{Speedup/thread e tempo/thread con l'i5, prima e seconda implementazione, contro password di lunghezza 6.}
\begin{subfigure}[t]{0.48\textwidth}
\resizebox{\textwidth}{!}{
	\begin{tikzpicture}
		\begin{axis}[
			scale only axis,
			width=\textwidth,
			height=\textheight/2,
			xtick=data,
			domain=1:4,
			ymajorgrids,
			xlabel={\# thread},
			ylabel={speedup},
        ]
			\pgfplotstableread{tests/i5_prima_6.dat}{\noh}
			\pgfplotstableread{tests/i5_prima_6h.dat}{\sih}
			\addplot[mark=o,red] 
			 table[x=thread,y=speedup]  {\noh};
			 \addlegendentry{No hyperthreading}
			\addplot[mark=o,blue] 
			 table[x=thread,y=speedup]  {\sih};
			 \addlegendentry{Hyperthreading}
			\addplot[dashed,blue] {x};
			\addplot[dashed] {2};
			\node[above] at (axis cs:1.5,2) {\# processori};
		\end{axis}
	\end{tikzpicture}
	}
	\caption{I implementazione}
\end{subfigure}
\qquad
\begin{subfigure}[t]{0.48\textwidth}
\resizebox{\textwidth}{!}{
	\begin{tikzpicture}
		\begin{axis}[
			scale only axis,
			width=\textwidth,
			height=\textheight/2,
			xtick=data,
			domain=1:4,
			ymajorgrids,
			xlabel={\# thread},
			ylabel={speedup},
        ]
			\pgfplotstableread{tests/i5_seconda_6.dat}{\noh}
			\pgfplotstableread{tests/i5_seconda_6h.dat}{\sih}
			\addplot[mark=o,red] 
			 table[x=thread,y=speedup]  {\noh};
			 \addlegendentry{No hyperthreading}
			\addplot[mark=o,blue] 
			 table[x=thread,y=speedup]  {\sih};
			 \addlegendentry{Hyperthreading}
			\addplot[dashed,blue] {x};
			\addplot[dashed] {2};
			\node[above] at (axis cs:1.5,2) {\# processori};
		\end{axis}
	\end{tikzpicture}
	}
	\caption{II implementazione}
\end{subfigure}
\\
\begin{subfigure}[b]{0.48\textwidth}
\resizebox{\textwidth}{!}{
	\begin{tikzpicture}
		\begin{axis}[
			scale only axis,
			width=\textwidth,
			height=\textheight/2,
			xtick=data,
			ymajorgrids,
			ymin=0.1,
			xlabel={\# thread},
			ylabel={tempo ($s$)},
		]
			\pgfplotstableread{tests/i5_prima_6.dat}{\pimone}
			\pgfplotstableread{tests/i5_prima_6h.dat}{\pimtwo}
			\addplot[mark=o,red] table[x=thread,y=tempo] {\pimone};
			 \addlegendentry{No hyperthreading}
			\addplot[mark=o,blue] table[x=thread,y=tempo] {\pimtwo};
			 \addlegendentry{Hyperthreading}
			\pgfplotstablegetelem{0}{tempo}\of{\pimone};
			\pgfmathsetmacro{\yn}{\pgfplotsretval};
    		\addplot[domain=1:4,dashed,samples=400] (x,{\yn/x});
			\addplot[domain=1:4,dashed] {\yn/2};
  		\end{axis}
	\end{tikzpicture}
	}
	\caption{I implementazione}
\end{subfigure}
\qquad
\begin{subfigure}[b]{0.48\textwidth}
\resizebox{\textwidth}{!}{
	\begin{tikzpicture}
		\begin{axis}[
			scale only axis,
			width=\textwidth,
			height=\textheight/2,
			xtick=data,
			ymajorgrids,
			ymin=0.1,
			xlabel={\# thread},
			ylabel={tempo ($s$)},
		]
			\pgfplotstableread{tests/i5_seconda_6.dat}{\pimone}
			\pgfplotstableread{tests/i5_seconda_6h.dat}{\pimtwo}
			\addplot[mark=o,red] table[x=thread,y=tempo] {\pimone};
			 \addlegendentry{No hyperthreading}
			\addplot[mark=o,blue] table[x=thread,y=tempo] {\pimtwo};
			 \addlegendentry{Hyperthreading}
			\pgfplotstablegetelem{0}{tempo}\of{\pimone};
			\pgfmathsetmacro{\yn}{\pgfplotsretval};
    		\addplot[domain=1:4,dashed,samples=400] (x,{\yn/x});
			\addplot[domain=1:4,dashed] {\yn/2};
  		\end{axis}
	\end{tikzpicture}
	}
	\caption{II implementazione}
\end{subfigure}
\end{figure}

\clearpage

\begin{figure*}[h]
	\caption{Speedup/thread su iMac, prima implementazione, contro password di lunghezza 5.}
\resizebox{\textwidth}{!}{
	\begin{tikzpicture}
		\begin{axis}[
			scale only axis,
			width=\textwidth,
			height=\textheight/1.1,
			xtick=data,
			domain=1:16,
			ymajorgrids,
			xlabel={\# thread},
			ylabel={speedup},
			ymax=8
        ]
			\pgfplotstableread{tests/imac_prima_51.dat}{\imacone}
			\pgfplotstableread{tests/imac_prima_52.dat}{\imactwo}
			\pgfplotstableread{tests/imac_prima_53.dat}{\imacthree}
			\addplot[mark=o,red] 
			 table[x=thread,y=speedup]  {\imacone};
			 \addlegendentry{Work unit = 1}
			\addplot[mark=o,blue] 
			 table[x=thread,y=speedup]  {\imactwo};
			 \addlegendentry{Work unit = 2}
			\addplot[mark=o,green]
			 table[x=thread,y=speedup]  {\imacthree};
			 \addlegendentry{Work unit = 3}
			\addplot[dashed,blue] {x};
			\addplot[dashed] {4};
			\node[above] at (axis cs:2,4) {\# processori};
		\end{axis}
	\end{tikzpicture}
	}
\end{figure*}

\clearpage

\begin{figure*}[h]
	\caption{Tempo/thread su iMac, prima implementazone, contro password di lunghezza 5.}
\resizebox{\textwidth}{!}{
	\begin{tikzpicture}
		\begin{axis}[
			scale only axis,
			width=\textwidth,
			height=\textheight/1.1,
			xtick=data,
			ymajorgrids,
			ymin=0.1,
			xlabel={\# thread},
			ylabel={tempo ($s$)},
		]
			\pgfplotstableread{tests/imac_prima_51.dat}{\pimone}
			\pgfplotstableread{tests/imac_prima_52.dat}{\pimtwo}
			\pgfplotstableread{tests/imac_prima_53.dat}{\pimthree}
			\addplot[mark=o,blue] table[x=thread,y=tempo] {\pimone};
			 \addlegendentry{Work unit = 1}
			\addplot[mark=o,red] table[x=thread,y=tempo] {\pimtwo};
			 \addlegendentry{Work unit = 2}
			\addplot[mark=o,green] table[x=thread,y=tempo] {\pimthree};
			 \addlegendentry{Work unit = 3}
			\pgfplotstablegetelem{0}{tempo}\of{\pimone};
			\pgfmathsetmacro{\yn}{\pgfplotsretval};
    		\addplot[domain=1:16,dashed,samples=400] (x,{\yn/x});
			\addplot[domain=1:16,dashed] {\yn/4};
  		\end{axis}
	\end{tikzpicture}
	}
\end{figure*}

\clearpage

\begin{figure}
\centering
\caption{Speedup/thread e tempo/thread della seconda implementazione, contro password di lunghezza 5, sul Raspberry Pi.}
\begin{subfigure}[t]{0.3\textwidth}
\resizebox{\textwidth}{!}{
	\begin{tikzpicture}
		\begin{axis}[
			scale only axis,
			width=\textwidth,
			xtick=data,
			domain=1:8,
			ymajorgrids,
			xlabel={\# thread},
			ylabel={speedup},
        ]
			\pgfplotstableread{tests/pi_seconda_2_51.dat}{\pimone}
			\pgfplotstableread{tests/pi_seconda_2_52.dat}{\pimtwo}
			\pgfplotstableread{tests/pi_seconda_2_53.dat}{\pimthree}
			\addplot[mark=o,blue] 
			 table[x=thread,y=speedup]  {\pimone};
			 \addlegendentry{Work unit = 1}
			\addplot[mark=o,red] 
			 table[x=thread,y=speedup]  {\pimtwo};
			 \addlegendentry{Work unit = 2}
			\addplot[mark=o,green]
			 table[x=thread,y=speedup]  {\pimthree};
			 \addlegendentry{Work unit = 3}
			\addplot[dashed,blue] {x};
			\addplot[dashed] {4};
			\node[above] at (axis cs:2,4) {\# processori};
		\end{axis}
	\end{tikzpicture}
	}
	\caption{JVM Oracle (1.7)}
\end{subfigure}
\qquad
\begin{subfigure}[t]{0.3\textwidth}
\resizebox{\textwidth}{!}{
	\begin{tikzpicture}
		\begin{axis}[
			scale only axis,
			width=\textwidth,
			xtick=data,
			domain=1:8,
			ymajorgrids,
			xlabel={\# thread},
			ylabel={speedup},
        ]
			\pgfplotstableread{tests/pi_seconda_4_51.dat}{\pimone}
			\pgfplotstableread{tests/pi_seconda_4_52.dat}{\pimtwo}
			\pgfplotstableread{tests/pi_seconda_4_53.dat}{\pimthree}
			\addplot[mark=o,blue] 
			 table[x=thread,y=speedup]  {\pimone};
			 \addlegendentry{Work unit = 1}
			\addplot[mark=o,red] 
			 table[x=thread,y=speedup]  {\pimtwo};
			 \addlegendentry{Work unit = 2}
			\addplot[mark=o,green]
			 table[x=thread,y=speedup]  {\pimthree};
			 \addlegendentry{Work unit = 3}
			\addplot[dashed,blue] {x};
			\addplot[dashed] {4};
			\node[above] at (axis cs:2,4) {\# processori};
		\end{axis}
	\end{tikzpicture}
	}
	\caption{JVM Oracle (1.8)}
\end{subfigure}
\qquad
\begin{subfigure}[t]{0.3\textwidth}
\resizebox{\textwidth}{!}{
	\begin{tikzpicture}
		\begin{axis}[
			scale only axis,
			width=\textwidth,
			xtick=data,
			domain=1:8,
			ymajorgrids,
			xlabel={\# thread},
			ylabel={speedup},
        ]
			\pgfplotstableread{tests/pi_seconda_3_51.dat}{\pimone}
			\pgfplotstableread{tests/pi_seconda_3_52.dat}{\pimtwo}
			\pgfplotstableread{tests/pi_seconda_3_53.dat}{\pimthree}
			\addplot[mark=o,blue] 
			 table[x=thread,y=speedup]  {\pimone};
			 \addlegendentry{Work unit = 1}
			\addplot[mark=o,red] 
			 table[x=thread,y=speedup]  {\pimtwo};
			 \addlegendentry{Work unit = 2}
			\addplot[mark=o,green]
			 table[x=thread,y=speedup]  {\pimthree};
			 \addlegendentry{Work unit = 3}
			\addplot[dashed,blue] {x};
			\addplot[dashed] {4};
			\node[above] at (axis cs:2,4) {\# processori};
		\end{axis}
	\end{tikzpicture}
	}
	\caption{JVM OpenJDK}
\end{subfigure}
\\
\begin{subfigure}[b]{0.3\textwidth}
\resizebox{\textwidth}{!}{
	\begin{tikzpicture}
		\begin{axis}[
			scale only axis,
			width=\textwidth,
			xtick=data,
			ymajorgrids,
			ymin=0.1,
			xlabel={\# thread},
			ylabel={tempo ($s$)},
		]
			\pgfplotstableread{tests/pi_seconda_2_51.dat}{\pimone}
			\pgfplotstableread{tests/pi_seconda_2_52.dat}{\pimtwo}
			\pgfplotstableread{tests/pi_seconda_2_53.dat}{\pimthree}
			\addplot[mark=o,blue] table[x=thread,y=tempo] {\pimone};
			 \addlegendentry{Work unit = 1}
			\addplot[mark=o,red] table[x=thread,y=tempo] {\pimtwo};
			 \addlegendentry{Work unit = 2}
			\addplot[mark=o,green] table[x=thread,y=tempo] {\pimthree};
			 \addlegendentry{Work unit = 3}
			\pgfplotstablegetelem{0}{tempo}\of{\pimone};
			\pgfmathsetmacro{\yn}{\pgfplotsretval};
    		\addplot[domain=1:8,dashed,samples=400] (x,{\yn/x});
			\addplot[domain=1:8,dashed] {\yn/4};
  		\end{axis}
	\end{tikzpicture}
	}
	\caption{JVM Oracle (1.7)}
\end{subfigure}
\qquad
\begin{subfigure}[b]{0.3\textwidth}
\resizebox{\textwidth}{!}{
	\begin{tikzpicture}
		\begin{axis}[
			scale only axis,
			width=\textwidth,
			xtick=data,
			ymajorgrids,
			ymin=0.1,
			xlabel={\# thread},
			ylabel={tempo ($s$)},
		]
			\pgfplotstableread{tests/pi_seconda_4_51.dat}{\pimone}
			\pgfplotstableread{tests/pi_seconda_4_52.dat}{\pimtwo}
			\pgfplotstableread{tests/pi_seconda_4_53.dat}{\pimthree}
			\addplot[mark=o,blue] table[x=thread,y=tempo] {\pimone};
			 \addlegendentry{Work unit = 1}
			\addplot[mark=o,red] table[x=thread,y=tempo] {\pimtwo};
			 \addlegendentry{Work unit = 2}
			\addplot[mark=o,green] table[x=thread,y=tempo] {\pimthree};
			 \addlegendentry{Work unit = 3}
			\pgfplotstablegetelem{0}{tempo}\of{\pimone};
			\pgfmathsetmacro{\yn}{\pgfplotsretval};
    		\addplot[domain=1:8,dashed,samples=400] (x,{\yn/x});
			\addplot[domain=1:8,dashed] {\yn/4};
  		\end{axis}
	\end{tikzpicture}
	}
	\caption{JVM Oracle (1.8)}
\end{subfigure}
\qquad
\begin{subfigure}[b]{0.3\textwidth}
\resizebox{\textwidth}{!}{
	\begin{tikzpicture}
		\begin{axis}[
			scale only axis,
			width=\textwidth,
			xtick=data,
			ymajorgrids,
			ymin=0.1,
			xlabel={\# thread},
			ylabel={tempo ($s$)},
		]
			\pgfplotstableread{tests/pi_seconda_3_51.dat}{\pimone}
			\pgfplotstableread{tests/pi_seconda_3_52.dat}{\pimtwo}
			\pgfplotstableread{tests/pi_seconda_3_53.dat}{\pimthree}
			\addplot[mark=o,blue] table[x=thread,y=tempo] {\pimone};
			 \addlegendentry{Work unit = 1}
			\addplot[mark=o,red] table[x=thread,y=tempo] {\pimtwo};
			 \addlegendentry{Work unit = 2}
			\addplot[mark=o,green] table[x=thread,y=tempo] {\pimthree};
			 \addlegendentry{Work unit = 3}
			\pgfplotstablegetelem{0}{tempo}\of{\pimone};
			\pgfmathsetmacro{\yn}{\pgfplotsretval};
    		\addplot[domain=1:8,dashed,samples=400] (x,{\yn/x});
			\addplot[domain=1:8,dashed] {\yn/4};
  		\end{axis}
	\end{tikzpicture}
	}
	\caption{JVM OpenJDK}
\end{subfigure}
\end{figure}

\clearpage

% HERE

\begin{figure}
	\caption{Speedup/thread e tempo/thread su Raspberry Pi, seconda implementazione, contro password di lunghezza 6.}
\begin{subfigure}[b]{0.48\textwidth}
\resizebox{\textwidth}{!}{
	\begin{tikzpicture}
		\begin{axis}[
			scale only axis,
			width=\textwidth,
			xtick=data,
			domain=1:8,
			ymajorgrids,
			xlabel={\# thread},
			ylabel={speedup},
        ]
			\pgfplotstableread{tests/pi_seconda_2_61.dat}{\pimone}
			\pgfplotstableread{tests/pi_seconda_4_61.dat}{\pimtwo}
			\addplot[mark=o,red] 
			 table[x=thread,y=speedup]  {\pimone};
			 \addlegendentry{1.7}
			\addplot[mark=o,blue] 
			 table[x=thread,y=speedup]  {\pimtwo};
			 \addlegendentry{1.8}
			\addplot[dashed,blue] {x};
			\addplot[dashed] {4};
			\node[above] at (axis cs:2,4) {\# processori};
		\end{axis}
	\end{tikzpicture}
	}
\end{subfigure}
\qquad
\begin{subfigure}[b]{0.48\textwidth}
\resizebox{\textwidth}{!}{
	\begin{tikzpicture}
		\begin{axis}[
			scale only axis,
			width=\textwidth,
			xtick=data,
			ymajorgrids,
			ymin=0.1,
			xlabel={\# thread},
			ylabel={tempo ($s$)},
		]
			\pgfplotstableread{tests/pi_seconda_2_61.dat}{\pimone}
			\pgfplotstableread{tests/pi_seconda_4_61.dat}{\pimtwo}
			\addplot[mark=o,red] table[x=thread,y=tempo] {\pimone};
			 \addlegendentry{1.7}
			\addplot[mark=o,blue] table[x=thread,y=tempo] {\pimtwo};
			 \addlegendentry{1.8}
			\pgfplotstablegetelem{0}{tempo}\of{\pimone};
			\pgfmathsetmacro{\yn}{\pgfplotsretval};
    		\addplot[domain=1:8,dashed,samples=400] (x,{\yn/x});
			\addplot[domain=1:8,dashed] {\yn/4};
  		\end{axis}
	\end{tikzpicture}
	}
\end{subfigure}
\end{figure}

\clearpage

\begin{figure*}[h]
	\caption{Speedup/thread su iMac, seconda implementazione, contro password di lunghezza 5.}
\resizebox{\textwidth}{!}{
	\begin{tikzpicture}
		\begin{axis}[
			scale only axis,
			width=\textwidth,
			height=\textheight/1.1,
			xtick=data,
			domain=1:16,
			ymajorgrids,
			xlabel={\# thread},
			ylabel={speedup},
			ymax=9
        ]
			\pgfplotstableread{tests/imac_seconda_51.dat}{\imacone}
			\pgfplotstableread{tests/imac_seconda_52.dat}{\imactwo}
			\pgfplotstableread{tests/imac_seconda_53.dat}{\imacthree}
			\pgfplotstableread{tests/imac_seconda_54.dat}{\imacfour}
			\addplot[mark=o,green] 
			 table[x=thread,y=speedup]  {\imacone};
			 \addlegendentry{Work unit = 1}
			\addplot[mark=o,blue] 
			 table[x=thread,y=speedup]  {\imactwo};
			 \addlegendentry{Work unit = 2}
			\addplot[mark=o,red]
			 table[x=thread,y=speedup]  {\imacthree};
			 \addlegendentry{Work unit = 3}
			\addplot[mark=o,orange]
			 table[x=thread,y=speedup]  {\imacfour};
			 \addlegendentry{Work unit = 4}
			\addplot[dashed,blue] {x};
			\addplot[dashed] {4};
			\node[above] at (axis cs:2,4) {\# processori};
		\end{axis}
	\end{tikzpicture}
	}
\end{figure*}

\clearpage

\begin{figure*}
	\caption{Tempo/thread su iMac, seconda implementazone, contro password di lunghezza 5.}
\resizebox{\textwidth}{!}{
	\begin{tikzpicture}
		\begin{axis}[
			scale only axis,
			width=\textwidth,
			height=\textheight/1.1,
			xtick=data,
			ymajorgrids,
			ymin=0.1,
			xlabel={\# thread},
			ylabel={tempo ($s$)},
		]
			\pgfplotstableread{tests/imac_seconda_51.dat}{\imacone}
			\pgfplotstableread{tests/imac_seconda_52.dat}{\imactwo}
			\pgfplotstableread{tests/imac_seconda_53.dat}{\imacthree}
			\pgfplotstableread{tests/imac_seconda_54.dat}{\imacfour}
			\addplot[mark=o,blue] table[x=thread,y=tempo] {\imacone};
			 \addlegendentry{Work unit = 1}
			\addplot[mark=o,red] table[x=thread,y=tempo] {\imactwo};
			 \addlegendentry{Work unit = 2}
			\addplot[mark=o,green] table[x=thread,y=tempo] {\imacthree};
			 \addlegendentry{Work unit = 3}
			\addplot[mark=o,orange] table[x=thread,y=tempo] {\imacfour};
			 \addlegendentry{Work unit = 4}
			\pgfplotstablegetelem{0}{tempo}\of{\imacone};
			\pgfmathsetmacro{\yn}{\pgfplotsretval};
    		\addplot[domain=1:16,dashed,samples=400] (x,{\yn/x});
			\addplot[domain=1:16,dashed] {\yn/4};
  		\end{axis}
	\end{tikzpicture}
	}
\end{figure*}

\clearpage

\begin{figure*}
	\caption{Speedup/thread su iMac, seconda implementazione, contro password di lunghezza 6.}
\resizebox{\textwidth}{!}{
	\begin{tikzpicture}
		\begin{axis}[
			scale only axis,
			width=\textwidth,
			height=\textheight/1.1,
			xtick=data,
			domain=1:16,
			ymajorgrids,
			xlabel={\# thread},
			ylabel={speedup},
			ymax=9
        ]
			\pgfplotstableread{tests/imac_seconda_61.dat}{\imacone}
			\pgfplotstableread{tests/imac_seconda_62.dat}{\imactwo}
			\pgfplotstableread{tests/imac_seconda_63.dat}{\imacthree}
			\pgfplotstableread{tests/imac_seconda_64.dat}{\imacfour}
			\addplot[mark=o,green] 
			 table[x=thread,y=speedup]  {\imacone};
			 \addlegendentry{Work unit = 1}
			\addplot[mark=o,blue] 
			 table[x=thread,y=speedup]  {\imactwo};
			 \addlegendentry{Work unit = 2}
			\addplot[mark=o,red]
			 table[x=thread,y=speedup]  {\imacthree};
			 \addlegendentry{Work unit = 3}
			\addplot[mark=o,orange]
			 table[x=thread,y=speedup]  {\imacfour};
			 \addlegendentry{Work unit = 4}
			\addplot[dashed,blue] {x};
			\addplot[dashed] {4};
			\node[above] at (axis cs:2,4) {\# processori};
		\end{axis}
	\end{tikzpicture}
	}
\end{figure*}

\clearpage

\begin{figure*}
	\caption{Tempo/thread su iMac, seconda implementazone, contro password di lunghezza 6.}
\resizebox{\textwidth}{!}{
	\begin{tikzpicture}
		\begin{axis}[
			scale only axis,
			width=\textwidth,
			height=\textheight/1.1,
			xtick=data,
			ymajorgrids,
			ymin=0.1,
			xlabel={\# thread},
			ylabel={tempo ($s$)},
		]
			\pgfplotstableread{tests/imac_seconda_61.dat}{\imacone}
			\pgfplotstableread{tests/imac_seconda_62.dat}{\imactwo}
			\pgfplotstableread{tests/imac_seconda_63.dat}{\imacthree}
			\pgfplotstableread{tests/imac_seconda_64.dat}{\imacfour}
			\addplot[mark=o,blue] table[x=thread,y=tempo] {\imacone};
			 \addlegendentry{Work unit = 1}
			\addplot[mark=o,red] table[x=thread,y=tempo] {\imactwo};
			 \addlegendentry{Work unit = 2}
			\addplot[mark=o,green] table[x=thread,y=tempo] {\imacthree};
			 \addlegendentry{Work unit = 3}
			\addplot[mark=o,orange] table[x=thread,y=tempo] {\imacfour};
			 \addlegendentry{Work unit = 4}
			\pgfplotstablegetelem{0}{tempo}\of{\imacone};
			\pgfmathsetmacro{\yn}{\pgfplotsretval};
    		\addplot[domain=1:16,dashed,samples=400] (x,{\yn/x});
			\addplot[domain=1:16,dashed] {\yn/4};
  		\end{axis}
	\end{tikzpicture}
	}
\end{figure*}

\clearpage

\begin{figure*}
	\caption{Speedup/thread su iMac, contro i file di benchmark.}
\resizebox{\textwidth}{!}{
	\begin{tikzpicture}
		\begin{axis}[
			scale only axis,
			width=\textwidth,
			height=\textheight/1.1,
			xtick=data,
			domain=1:16,
			ymajorgrids,
			xlabel={\# thread},
			ylabel={speedup},
			ymax=8
        ]
			\pgfplotstableread{tests/imac_benchmark_easy.dat}{\bencheasy}
			\pgfplotstableread{tests/imac_benchmark_medium.dat}{\benchmedium}
			\pgfplotstableread{tests/imac_benchmark_hard.dat}{\benchhard}
			\addplot[mark=o,green] 
			 table[x=thread,y=speedup]  {\bencheasy};
			 \addlegendentry{easy.txt}
			\addplot[mark=o,blue] 
			 table[x=thread,y=speedup]  {\benchmedium};
			 \addlegendentry{medium.txt}
			\addplot[mark=o,red]
			 table[x=thread,y=speedup]  {\benchhard};
			 \addlegendentry{hard.txt}
			\addplot[dashed,blue] {x};
			\addplot[dashed] {4};
			\node[above] at (axis cs:2,4) {\# processori};
		\end{axis}
	\end{tikzpicture}
	}
\end{figure*}


\end{document}
